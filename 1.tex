\documentclass[12pt,a4paper]{article} % 国赛官方要求字号

%---------- 基础宏包 ----------
\usepackage[UTF8]{ctex}          % 中文支持
\usepackage{amsmath,amssymb}     % 数学公式
\usepackage{graphicx}            % 插图支持
\usepackage{booktabs}            % 专业三线表
\usepackage{algorithm}           % 算法伪代码
\usepackage{algpseudocode}       % 算法结构
\usepackage{hyperref}            % 超链接(自动生成目录)
\usepackage{array} % 表格增强
\usepackage[table]{xcolor} % 为了加颜色或加框
\usepackage{float}
\usepackage{tcolorbox}
\usepackage{threeparttable} % 用于表格注释
\usepackage{diagbox}  % 斜线单元格支持
\tcbuselibrary{skins}
\tcbuselibrary{breakable}

% 定义引用环境样式
\newtcolorbox{markdownquote}{
  enhanced,
  breakable,
  left = 5pt,
  right = 0pt,
  top = 0pt,
  bottom = 0pt,
  colback = white,
  colframe = gray!30, % 竖线颜色
  arc = 0pt,
  boxrule = 0pt,
  leftrule = 3pt, % 竖线宽度
  coltext = gray!70, % 文字颜色
  fontupper = \normalfont,
  before skip = 10pt,
  after skip = 10pt,
}
\newcommand{\ps}[1]{%
    \par\vspace{1em}% 与上文空出一定距离
    \noindent\textbf{PS:}~#1\par% 以粗体的"PS:"开头,然后接上具体的补充内容
}
\renewcommand{\algorithmicrequire}{\textbf{输入:}} 
\renewcommand{\algorithmicensure}{\textbf{输出:}} 

\begin{document}

\tableofcontents
\newpage

%以下是笔记内容
\section{层次分析法\protect[AHP\protect]}
\begin{markdownquote}
    画层次结构图,构造判断矩阵,一致性检验,计算权重向量,综合评价。
\end{markdownquote}
\subsection{构造层次评价模型}
分为目标层、准则层和方案层。
目标层是我们要解决的问题,准则层是评价标准,方案层是备选方案。
\textbf{需要在论文中给出层次结构图。}
\begin{figure}[htbp] % 浮动体环境,参数建议位置:here, top, bottom, page
  \centering % 图片居中
  \includegraphics[width=0.6\textwidth]{层次结构图.png} % 图片宽度为文本宽度的50%
  \caption{层次结构图} % 图片标题
  \label{fig:sample} % 标签,方便引用
\end{figure}
\subsection{构造判断矩阵}
以准则层为例,可以根据重要性进行两两比较,构造判断矩阵。理论上对应的分数需要进行问卷调查判断重要程度,但往往采用自己编或者问AI的方式来给出分数。
然后记住,以上图为例,则一共会有6个判断矩阵。分别为1个5$\times$5的准则层判断矩阵和5个4$\times$4的方案层判断矩阵。
\begin{figure}[H] % 浮动体环境,参数建议位置:here, top, bottom, page
  \centering % 图片居中
  \includegraphics[width=1\textwidth]{2.png} % 图片宽度为文本宽度的50%
    \caption{判断矩阵示例} % 图片标题
  \label{fig:judgment_matrix} % 标签,方便引用
\end{figure}
\subsection{一致性检验}
对以上所有判断矩阵计算一致性指标CI和一致性比率CR。
\begin{equation}
CI = \frac{\lambda_{max} - n}{n - 1}
\end{equation}
\begin{equation}
CR = \frac{CI}{RI}
\end{equation}
其中,$\lambda_{max}$为判断矩阵的最大特征值,$n$为判断矩阵的阶数,$RI$为随机一致性指标(可以查表)。
如果CR小于0.1,则判断矩阵一致性通过,否则需要调整判断矩阵。
RI的值可以通过查表得到,具体如下:
\begin{table}[H]
    \centering
    \begin{threeparttable}
    \caption{随机一致性指标 RI 表}
    \begin{tabular}{c|ccccccccccc}
    \toprule % 顶部粗横线,若不用 booktabs 包,可替换为 \hline
    $n$ & 1 & 2 & 3 & 4 & 5 & 6 & 7 & 8 & 9 & 10 & 11 \\
    \midrule % 中间横线,若不用 booktabs 包,可替换为 \hline
    $RI$ & 0 & 0 & 0.58 & 0.90 & 1.12 & 1.24 & 1.32 & 1.41 & 1.45 & 1.49 & 1.51 \\
    \bottomrule % 底部粗横线,若不用 booktabs 包,可替换为 \hline
    \end{tabular}
    \begin{tablenotes}
            \small % 注释文字使用小字号
            \item[*] \textcolor{red}{注:需要体现在论文中!!!}
            \item[] 大于0.1尽量往一致性上靠
            \item[] 数据来源:Saaty T. L. (1980). The Analytic Hierarchy Process.
        \end{tablenotes}
    \end{threeparttable}
    \label{tab:ri_table}
\end{table}
\subsection{计算权重向量}
计算权重向量的方式有两种:特征值法和算术平均法。
\subsubsection{特征值法}
\[
\boldsymbol{A} = \begin{bmatrix}
1 & 2 & 5 \\
1/2 & 1 & 2 \\
1/5 & 1/2 & 1
\end{bmatrix}
\xrightarrow{\text{计算特征值和特征向量}} 
\lambda_1 = 3.0055,\ \lambda_2 = -0.0028
\]

\[
\xrightarrow{\text{一致性检验}} 
CI = 0.0053 < 0.1,\ \text{通过一致性检验} 
\xrightarrow{\text{最大特征值对应的特征向量}} 
\boldsymbol{\alpha} = \begin{pmatrix}
-0.8902 \\
-0.4132 \\
-0.1918
\end{pmatrix}
\]

\[
\xrightarrow{\text{归一化}} 
\boldsymbol{\omega} = \begin{pmatrix}
0.5954 \\
0.2764 \\
0.1283
\end{pmatrix} 
\ 
\]
\subsubsection{算术平均法}
对于判断矩阵
\[
\boldsymbol{A} = \begin{bmatrix}
a_{11} & a_{12} & \cdots & a_{1n} \\
a_{21} & a_{22} & \cdots & a_{2n} \\
\vdots & \vdots & \ddots & \vdots \\
a_{n1} & a_{n2} & \cdots & a_{nn}
\end{bmatrix},
\]
先将其归一化,再将归一化的矩阵按列相加,并将每个元素除以 \( n \) 得到权重向量,即
\[
\omega_i = \frac{1}{n} \sum_{j=1}^{n} \frac{a_{ij}}{\sum_{k=1}^{n} a_{kj}} \quad (i = 1, 2, \cdots, n).
\]
\begin{markdownquote}
    若是标准的一致性矩阵,则不用求平均值,计算单列即可。且按行,按列影响不大。
\end{markdownquote}

\subsection{层次总排序和一致性检验}
将各层的权重向量进行综合,得到最终的综合评价结果。
\begin{equation}
S = \sum_{i=1}^{n} w_i \cdot x_i
\end{equation}
\begin{table}[!ht]
\centering
\caption{层次总排序一致性检验结果(所有准则层均为3阶矩阵,RI=0.58)}
\begin{tabular}{|c|c|c|c|c|c|c|}
\hline
& \textbf{权重} & \textbf{苏杭} & \textbf{北戴河} & \textbf{桂林} & \textbf{$CI$} & \textbf{$CR/\lambda_{\text{max}}$} \\ \hline
\textbf{景色}   & \cellcolor[HTML]{FDDBC7}0.2910 & \cellcolor[HTML]{FDDBC7}0.5954 & \cellcolor[HTML]{FDDBC7}0.2764 & \cellcolor[HTML]{FDDBC7}0.1283 & \cellcolor[HTML]{FDDBC7}0.0028 & \cellcolor[HTML]{FDDBC7}0.0053/3.0055 \\ \hline
\textbf{花费}   & \cellcolor[HTML]{FDE9D9}0.5022 & \cellcolor[HTML]{FDE9D9}0.4434 & \cellcolor[HTML]{FDE9D9}0.3874 & \cellcolor[HTML]{FDE9D9}0.1692 & \cellcolor[HTML]{FDE9D9}0.0091 & \cellcolor[HTML]{FDE9D9}0.0176/3.0183 \\ \hline
\textbf{饮食}   & \cellcolor[HTML]{D9EAD3}0.0648 & \cellcolor[HTML]{D9EAD3}0.5396 & \cellcolor[HTML]{D9EAD3}0.2970 & \cellcolor[HTML]{D9EAD3}0.1634 & \cellcolor[HTML]{D9EAD3}0.0046 & \cellcolor[HTML]{D9EAD3}0.0088/3.0092 \\ \hline
\textbf{男女比例} & \cellcolor[HTML]{D9D9D9}0.1420 & \cellcolor[HTML]{D9D9D9}0.6267 & \cellcolor[HTML]{D9D9D9}0.2797 & \cellcolor[HTML]{D9D9D9}0.0936 & \cellcolor[HTML]{D9D9D9}0.0429 & \cellcolor[HTML]{D9D9D9}0.0825/3.0858 \\ \hline
\textbf{最终得分} & \cellcolor[HTML]{FDE9D9}\diagbox[dir=SE,width=8em,height=2em]{}{} & \cellcolor[HTML]{FDE9D9}0.5199 & \cellcolor[HTML]{FDE9D9}0.3339 & \cellcolor[HTML]{FDE9D9}0.1462 & \multicolumn{2}{c|}{\cellcolor[HTML]{FDE9D9}} \\ \hline
\end{tabular}
\end{table}

\begin{equation}
\begin{aligned}
CR &= \frac{\sum_{i=1}^{4} a_i CI_i}{RI} \\
   &= \frac{0.2910 \times 0.0028 + 0.5022 \times 0.0091 + 0.0648 \times 0.0046 + 0.1420 \times 0.0429}{0.58} \\
   &= 0.0203 < 0.1
\end{aligned}
\end{equation}
由于 \( CR < 0.1 \),层次总排序通过一致性检验,结果可信
\subsection{范例}
\begin{figure}[H]
    \centering
    \includegraphics[width=1.2\textwidth]{大佬.png}
    \caption{AHP示例}
    \label{fig:ahp_example}
\end{figure}

\end{document}
